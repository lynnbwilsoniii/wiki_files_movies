\documentclass[pdftex,letterpaper,11pt]{article}

%%
%%  Graphics packages
%%
\pdfoutput=1
\usepackage[dvips,pdftex]{graphicx}
%%
%%  Math/Symbol packages
%%
\usepackage{amsmath}   %%  contains advanced math extensions
\usepackage{latexsym}  %%  adds other symbols in to be used in math mode
\usepackage{amssymb}   %%  adds new symbols in to be used in math mode
%%
%%  Page Formatting packages
%%
%%  Define page size and margin sizes
\usepackage[top=1.0in,bottom=1.5in, left=1.5in, right=1.5in]{geometry}
\linespread{1.00}         %%  Set line spacing to single space
\usepackage{indentfirst}  %%  Force first paragraph to indent
%%
%%  Referencing packages
%%
\usepackage{hyperref}     %%  Allow internal and external references
%%
%%  Reference Footnotes
%%
\usepackage{fmtcount}  %%  Allow user to keep track of counters/labels
%\fmtcountsetoptions{fmtord=level}  %%  change from superscript to [#]
\renewcommand{\fmtord}[1]{}  %%  remove st, nd, rd, th, etc. from ordinal suffixes

%%%%%%####################################################################################
%%%%%%%%  Begin the document
%%%%%%####################################################################################
\begin{document}

%%----------------------------------------------------------------------------------------
%%  Section:  First Section
%%----------------------------------------------------------------------------------------
\section{\label{sec:First}First Section}

\indent  This file shows an example of how to use the \textbf{\textit{fmtcount}} package to reference a footnote\footnote{This is the 1st footnote.}\storeordinalnum{ftn:SMfootnote}{\value{footnote}}.  We save the identifier by using the \\
\verb+\storeordinalnum{label}{number}+ \\
command\footnote{The \textbf{\textit{fmtcount}} package and documentation are included in the CTAN distribution.}, where \textit{label} is similar to the label taken by the \verb+\ref{label}+ command and \textit{number} is the counter value associated with the footnote retrieved by using \verb+\value{footnote}+.  The actual command, \verb+\FMCuse{label}+, used to reference \textit{label} is discussed in Section \ref{sec:Reference}.  

\indent  Note that it is often useful to create a prefix for each \textit{label} associated with the type of environment/command it references.  A common set of prefixes\footnote{These specific examples are relied upon by the \textbf{\textit{fancyref}} package.} are:  \emph{fig:} for figures; \emph{tab:} for tables; \emph{sec:} for sections; \emph{eq:} for equations; \emph{lst:} for code listing; and \emph{itm:} for enumerated lists.  I often use the following as well:  \emph{subsec:} for subsections; \emph{subsubsec:} for subsubsections; \emph{app:} for appendices; and \emph{ftn:} for footnotes.

%%----------------------------------------------------------------------------------------
%%  Section:  Second Section
%%----------------------------------------------------------------------------------------
\section{\label{sec:Second}Second Section}
\indent  Now we wish to reference the first footnote at some later point in the document, but we have created other footnotes\footnote{This is the 3rd footnote.} and sections (see Section \ref{sec:First}) etc.

\clearpage
%%----------------------------------------------------------------------------------------
%%  Section:  Reference First Footnote
%%----------------------------------------------------------------------------------------
\section{\label{sec:Reference}Reference First Footnote}

\noindent  Now we will reference the first footnote\footnotemark[\FMCuse{ftn:SMfootnote}] using \\

\noindent  \verb+\footnotemark[\FMCuse{label}]+ \\

\noindent  where \\

\noindent  \verb+\FMCuse{label}+ \\

\noindent  is part of the \textbf{\textit{fmtcount}} package and \\

\noindent  \verb+\footnotemark[number]+ \\

\noindent  is a standard part of the \LaTeX\ distribution.  \\

\indent  Notice that this did not print the footnote on the current page\footnote{I have found this useful for journal articles such as in \emph{Physical Review Letters} where the \emph{REVTeX} package causes footnotes to act similar to cited references.}.  The operations simply produced the footnote mark associated with the \textit{label} we defined in Section \ref{sec:First}.  

%%%%%%####################################################################################
%%%%%%%%  End the document
%%%%%%####################################################################################
\end{document}
